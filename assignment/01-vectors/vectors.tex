\documentclass[12pt]{article}

\usepackage[utf8]{inputenc}
\usepackage{geometry}
\geometry{
    a4paper,
    total={170mm,257mm},
    left=25mm,
    right=25mm,
    top=25mm,
    bottom=25mm,
}
\usepackage{multicol}
\usepackage[font=small,labelfont=bf]{caption}
\setlength{\columnsep}{0.25cm}
\usepackage[inline]{enumitem}
\usepackage{amssymb}
\usepackage{xcolor}
\usepackage{mathtools} 
\setlength{\parindent}{0em}
\setlength{\parsep}{0em}
\usepackage{tikz}
\setlength{\parskip}{0em}
\usetikzlibrary{decorations.pathmorphing,patterns}
\usepackage[american,cuteinductors]{circuitikz}
\usepackage{hyperref}
\usetikzlibrary{shapes,arrows,circuits,calc,babel}
% Definition of blocks:
\tikzset{%
  block/.style    = {draw, thick, rectangle, minimum height = 3em,
    minimum width = 3em},
  sum/.style      = {draw, circle, node distance = 2cm}, % Adder
  input/.style    = {coordinate}, % Input
  output/.style   = {coordinate} % Output
}
% Defining string as labels of certain blocks.
\newcommand{\suma}{\Large$+$}
\newcommand{\inte}{$\displaystyle \int$}
\newcommand{\derv}{\huge$\frac{d}{dt}$}

\def\mf{\ensuremath\mathbf}
\def\mb{\ensuremath\mathbb}
\def\mc{\ensuremath\mathcal}
\def\lp{\ensuremath\left(}
\def\rp{\ensuremath\right)}
\def\lv{\ensuremath\left\lvert}
\def\rv{\ensuremath\right\rvert}
\def\lV{\ensuremath\left\lVert}
\def\rV{\ensuremath\right\rVert}
\def\lc{\ensuremath\left\{}
\def\rc{\ensuremath\right\}}
\def\ls{\ensuremath\left[}
\def\rs{\ensuremath\right]}
\def\bmx{\ensuremath\begin{bmatrix*}[r]}
\def\emx{\ensuremath\end{bmatrix*}}
\def\bmxc{\ensuremath\begin{bmatrix*}[c]}
\def\emxc{\ensuremath\end{bmatrix*}}
% \def\t{\lp t\rp}
% \def\k{\ls k\rs}

\newcommand{\demoex}[2]{\onslide<#1->\begin{color}{black!60} #2 \end{color}}
\newcommand{\demoexc}[3]{\onslide<#1->\begin{color}{#2} #3 \end{color}}
\newcommand{\anim}[3]{\onslide<#1->{\begin{color}{#2!60} #3 \end{color}}}
\newcommand{\ct}[1]{\lp #1\rp}
\newcommand{\dt}[1]{\ls #1\rs}

% \renewcommand{\familydefault}{\sfdefault}

\begin{document}
\begin{center}
\begin{large}
\textbf{Applied Linear Algebra in Data Analysis}\\
\vspace{0.1cm}
\end{large}
\textbf{Vectors Assignment}
\end{center}
\hrule
\vspace{1em}

\begin{large}
    \textbf{Marks: 30}
\end{large}

Please upload the solutions to the programming questions on this link: \href{https://forms.gle/bZWAFWjm9yRcwkEa9}{\underline{Assignment 1}}

\begin{enumerate}
    \item Is this set of vectors $\left\{\begin{bmatrix}2 \\ 1 \\ 1\end{bmatrix}, \begin{bmatrix}0 \\ 0\\ 0\end{bmatrix}, \begin{bmatrix}0 \\ 1 \\ 0\end{bmatrix}\right\}$ independent? 
    
    Explain your answer. \textbf{[Marks: 2]}

    \item Consider a set of finite duration discrete-time real signals \textbf{[Marks: 2]}
    \[ X_N = \lc x\dt{n} \big\vert x\dt{n} \in \mb{R}, \,\forall 0 \leq n \leq N-1\rc \]
    
    Does this set form a vector space? Explain your answer. Would $X_N$ still be a vector space if the signals were binary signals? i.e. $x\dt{n} \in \mb{B}$, where $\mb{B} = \lc 0, 1\rc$ with the binary addition and multiplication operations defined as the following,

    \begin{center}
        \begin{minipage}[h]{.45\textwidth}
            \centering
            \begin{tabular}{|c|c|c|c|}
                \hline
                $a$ & $b$ & $a+b$ & $a \times b$ \\ \hline
                0 & 0 & 0   & 0     \\ \hline
                0 & 1 & 1   & 0     \\ \hline
                1 & 0 & 1   & 0     \\ \hline
                1 & 1 & 0   & 1     \\ \hline
            \end{tabular}\\
            \captionof{table}{\small{Addition and Multiplication operation for binary numbers.}}
        \end{minipage}
    \end{center}
        
    \item Prove the following for $\mf{x}, \mf{y} \in \mathbb{R}^n$,
    \begin{enumerate}
        \item {\small \textbf{Triangle Inequality}}: \textbf{[Marks: 1]}
        \[ \left\lVert \mf{x} + \mf{y}\right\rVert \leq \left\lVert \mf{x}\right\rVert + \left\lVert \mf{y}\right\rVert \]
        \item {\small \textbf{Backward Triangle Inequality}}: \textbf{[Marks: 1]}
        \[ \left\lVert \mf{x} - \mf{y}\right\rVert  \geq \left\lvert \left\lVert \mf{x}\right\rVert - \left\lVert \mf{y}\right\rVert \right\rvert \]
        \item {\small \textbf{Parallelogram Identity}}: \textbf{[Marks: 1]}
        \[ \frac{1}{2} \left(\left\lVert \mf{x} + \mf{y}\right\rVert^2 + \left\lVert \mf{x} - \mf{y}\right\rVert^2 \right) = \left\lVert \mf{x}\right\rVert^2 + \left\lVert \mf{y}\right\rVert^2 \]
    \end{enumerate}

    \item Consider a set of vectors $\mf{x}, \mf{y} \in \mathbb{R}^n$. When is $\left\lVert \mf{x} - \mf{y} \right\rVert = \left\lVert \mf{x} + \mf{y}\right\rVert$? What can you say about the geometry of the vectors $\mf{x},\,\mf{y},\,\mf{x} - \mf{y}$ and $\mf{x} + \mf{y}$? \textbf{[Marks: 2]}
    
    \item If $S_1, S_2 \subseteq V$ are subspaces of $V$ then, is $S_1 \cap S_2$ a subspace? Demonstrate your answer. \textbf{[Marks: 2]}
    
    \item Prove that the sum of two subspaces $S_1, S_2 \subseteq V$ is a subspace. \textbf{[Marks: 1]}
    
    \item Consider a vector $\mf{v} = \begin{bmatrix*}v_1 & v_2 & \cdots & v_n\end{bmatrix*}^\top$. Express the following in-terms of inner product between a constant vector $\mf{u}$ and the given vector $\mf{v}$, and in each case specify the vector $\mf{u}$. \textbf{[Marks: 3]}
    \begin{enumerate}
        \item $\sum_{i=1}^n v_i$
        \item $\frac{1}{n}\sum_{i=1}^n v_i$
        \item $\frac{1}{5}\sum_{i=3}^5 v_i$
    \end{enumerate} 
    
    \item Which of the following are linear functions of $\left\{x_1, x_2, \ldots,x_n\right\}$? \textbf{[Marks: 3]}
    \begin{enumerate}
        \item $\min_i \left\{x_i\right\}_{i=1}^{n}$
        \item $\left(\sum_{i=1}^n x_i^2\right)^{1/2}$
        \item $x_6$
    \end{enumerate}
    
    \item Consider a linear function $f: \mathbb{R}^n \rightarrow \mathbb{R}$. Prove that every linear function of this form can be represented in the following form. \textbf{[Marks: 2]}
    \[ y = f\left(\mf{x}\right) = \mf{w}^\top\mf{x} = \sum_{i=1}^{n}w_ix_i, \,\,\,\,\, \mf{x}, \mf{w} \in \mathbb{R}^n \]

    \item An \textit{affine} function $f$ is defined as the sum of a linear function and a constant. It can in general be represented in the form, 
    \[ y = f\left(\mf{x}\right) = \mf{w}^\top\mf{x} + \beta, \,\,\,\,\, \mf{x}, \mf{w} \in \mathbb{R}^n, \, \beta \in \mathbb{R} \]
    Prove that affine functions are not linear. Prove that any affine function can be represented in the form $\mf{w}^\top\mf{x} + \beta$. \textbf{[Marks: 2]}

    % \item Consider a function $f: \mathbb{R}^3 \rightarrow \mathbb{R}$, such that,
    % \[ f\left(\begin{bmatrix*}[r]1\\0\\0\end{bmatrix*}\right) = 2; \,\,
    % f\left(\begin{bmatrix*}[r]0\\1\\0\end{bmatrix*}\right) = -3; \,\,
    % f\left(\begin{bmatrix*}[r]0\\0\\1\end{bmatrix*}\right) = 1; \,\,\]
    % Can you determine the following values of $f\left(\mf{x}\right)$, if you are told that $f$ is linear?
    % \[ f\left(\begin{bmatrix*}[r]2\\2\\-2\end{bmatrix*}\right) = ?; \,\,\,
    % f\left(\begin{bmatrix*}[r]-1\\2\\0\end{bmatrix*}\right) = ?; \,\,\,
    % f\left(\begin{bmatrix*}[r]0.5\\0.6\\-0.1\end{bmatrix*}\right) = ?; \,\,\,\]
    % Can you find out these values if you are told that $f$ is affine?
    
    % \item For the previous question, (a) assume that $f$ is linear and find out $\mf{w} \in \mathbb{R}^3$, such that $f\left(\mf{x}\right) = \mf{w}^\top\mf{x}$; and (b) assume $f$ is affine and find out $\mf{w}, \beta$ such that $f\left(\mf{x}\right) = \mf{w}^\top\mf{x} + \beta$.

    % \item Consider the weighted norm of vector $\mf{v}$, defined as,
    % \[ \lVert \mf{v} \rVert_{\mf{w}}^2 = \sum_{i=1}^{n}w_iv_i^2; \quad \mf{w}=\begin{bmatrix*} w_1 & w_2 & \cdots & w_n\end{bmatrix*}^\top, w_i > 0 \,\, \forall i \]
    % Is this a valid norm?\\

    % \item Prove that the following modified version of the Cauchy-Bunyakovski-Schwartx Inequality is true.
    % \[ \left\lvert \sum_{i=1}^{n} u_i v_i w_i \right\rvert \leq \left\lVert \mf{u}\right\rVert_\mf{w} \left\lVert \mf{v}\right\rVert_\mf{w} \]

    \item Consider a basis $B = \left\{\mf{b}_i\right\}_{i=1}^{n}$ of $\mb{R}^n$. Let the vectors $\mf{x}$ and $\mf{x}_b$ be the representations in the standard and $B$ basis respectively. 
    \[ \mf{x} = \begin{bmatrix*}x_1\\x_2\\\vdots\\x_n\end{bmatrix*} = \sum_{i=1}^{n}x_i\mf{e}_i \,\,\,\,\, \text{and} \,\,\,\,\, \mf{x}_b =  \begin{bmatrix*}x_{b1}\\x_{b2}\\\vdots\\x_{bn}\end{bmatrix*} = \sum_{i=1}^{n}x_{bi}\mf{b}_i \]

    Evaluate the $\left\lVert \mf{x}\right\rVert_2^2$ and $\left\lVert \mf{x}_b\right\rVert_2^2$. Determine what happens to $\left\lVert \mf{x}_b\right\rVert_2^2$ under the following conditions on the basis vectors: \textbf{[Marks: 2]}
    \begin{enumerate}
        \item $\left\lVert \mf{b}_i\right\rVert = 1, \forall i$
        \item $\left\lVert \mf{b}_i^\top\mf{b}_j\right\rVert = \begin{cases}
        1 & i = j\\
        0 & i \neq j
        \end{cases}$
    \end{enumerate}

    \item \textcolor{blue}{\textbf{[Programming]}} Consider a set of measurements made from adult male subjects, where their height, weight and BMI (body mass index) were recorded and stored as vectors of length three; the first element is the height in $cm$, second is the weight in $Kg$, and the last is the BMI. Consider the following four subjects,
    \[ \mf{s}_1 =  \begin{bmatrix*}[r]167\\102\\36.6\end{bmatrix*}; \,\,
    \mf{s}_2 =  \begin{bmatrix*}[r]180\\87\\26.9\end{bmatrix*} \]
    \[ \mf{s}_3 =  \begin{bmatrix*}[r]177\\78\\24.9\end{bmatrix*}; \,\,
    \mf{s}_4 =  \begin{bmatrix*}[r]152\\76\\32.9\end{bmatrix*} \]

    You can use the distance between these vectors $\left\lVert \mf{s}_i - \mf{s}_j\right\rVert_2$ as a measure of the similiarity between the four subjects. Generate a $4 \times 4$ table comparing the distance of each subject with respect to another subject; the diagonal elements of this table will be zero, and it will be symmetric about the main diagonal. 

    (a) Based on this table, how do the different subjects compare to each other?  \textbf{[Marks: 1]}

    (b) How do the similarities change if the height had been measured in $m$ instead of $cm$? Can you explain this difference?  \textbf{[Marks: 1]}

    (c) Consider the weighted norm presented in one of the earlier problems.  
    \[ \left\lVert x \right\rVert_\mf{w} = \left(w_1 x_1^2 + w_2 x_2^2 + \cdots + w_n x_n^2\right)^{\frac{1}{2}} \]
    Will this fix the problem? What would be a good choice for $\mf{w}$ to address the problems with comparing distance between vectors due to unit change?\textbf{[Marks: 2]}

    (d) Can the angle between two vectors be used as a measure of similarity between vectors? Does this suffer from the problem of $\left\lVert \mf{x} \right\rVert_2$?  \textbf{[Marks: 2]}
\end{enumerate}

\end{document}