\documentclass[9pt]{article}

\usepackage[utf8]{inputenc}
\usepackage{geometry}
\geometry{
    a4paper,
    total={170mm,257mm},
    left=15mm,
    right=15mm,
    top=20mm,
    bottom=20mm,
}
\usepackage{multicol}
\usepackage[font=small,labelfont=bf]{caption}
\setlength{\columnsep}{0.25cm}
\usepackage[inline]{enumitem}
\usepackage{amssymb}
\usepackage{xcolor}
\usepackage{mathtools} 
\setlength{\parindent}{0em}
\setlength{\parsep}{0em}
\usepackage{tikz}
\setlength{\parskip}{0em}
\usetikzlibrary{decorations.pathmorphing,patterns}
\usepackage[american,cuteinductors]{circuitikz}
\usetikzlibrary{shapes,arrows,circuits,calc,babel}
% Definition of blocks:
\tikzset{%
  block/.style    = {draw, thick, rectangle, minimum height = 3em,
    minimum width = 3em},
  sum/.style      = {draw, circle, node distance = 2cm}, % Adder
  input/.style    = {coordinate}, % Input
  output/.style   = {coordinate} % Output
}
% Defining string as labels of certain blocks.
\newcommand{\suma}{\Large$+$}
\newcommand{\inte}{$\displaystyle \int$}
\newcommand{\derv}{\huge$\frac{d}{dt}$}

\def\mf{\ensuremath\mathbf}
\def\mb{\ensuremath\mathbb}
\def\mc{\ensuremath\mathcal}
\def\lp{\ensuremath\left(}
\def\rp{\ensuremath\right)}
\def\lv{\ensuremath\left\lvert}
\def\rv{\ensuremath\right\rvert}
\def\lV{\ensuremath\left\lVert}
\def\rV{\ensuremath\right\rVert}
\def\lc{\ensuremath\left\{}
\def\rc{\ensuremath\right\}}
\def\ls{\ensuremath\left[}
\def\rs{\ensuremath\right]}
\def\bmx{\ensuremath\begin{bmatrix*}[r]}
\def\emx{\ensuremath\end{bmatrix*}}
\def\bmxc{\ensuremath\begin{bmatrix*}[c]}
\def\emxc{\ensuremath\end{bmatrix*}}
% \def\t{\lp t\rp}
% \def\k{\ls k\rs}

\newcommand{\demoex}[2]{\onslide<#1->\begin{color}{black!60} #2 \end{color}}
\newcommand{\demoexc}[3]{\onslide<#1->\begin{color}{#2} #3 \end{color}}
\newcommand{\anim}[3]{\onslide<#1->{\begin{color}{#2!60} #3 \end{color}}}
\newcommand{\ct}[1]{\lp #1\rp}
\newcommand{\dt}[1]{\ls #1\rs}

\renewcommand{\familydefault}{\sfdefault}

\begin{document}
\begin{center}
\begin{Large}
\textbf{Linear Systems: Eigenvalues and Eigenvectors Assignment}
\end{Large}
\end{center}
\vspace{0.2cm}

\begin{multicols}{2}

\begin{enumerate}[resume]
    \item Explain why an eigenvector cannot be associated with two eigenvalues.

    \item What are the eigenspaces associated with the diagonal matrix $\mf{D} = \mathrm{diag}\lp d_1, d_2, \ldots d_n\rp$?

    \item If a matrix $\mf{A}$ has zero as one of its eigenvalues, explain why $\mf{A}$ must be singular.

    \item For a matrix $\mf{A}$ with eigenvalues $\lc\lambda_i\rc_{i=1}^{n}$, verify for the following matrices that $\Pi_{i=1}^n\lambda_i = \det \lp\mf{A}\rp$ and $\sum_{i=1}^n \lambda_i = trace\lp\mf{A}\rp$.
    \begin{enumerate}
        \item $\bmx 1 & 1 \\ 2 & 1\emx$
        \item $\bmx 1 & 0 & -1 \\ -1 & 1 & 0 \\ 2 & 1 & 1\emx$
        \item $\bmx 1 & 1 \\ 0 & 1\emx$
        \item $\frac{1}{5}\bmx 1 \\ 0 \\ 2 \emx \bmx 1 & 0 & 2\emx$
    \end{enumerate}

    \item Let $\lc\lambda_i, \mf{v}_i\rc_{i=1}^n$ be the eigenpairs of a matrix $\mf{A}$. Then prove that,
    \begin{enumerate}
        \item $\lc\lambda_i^k, \mf{v}_i\rc_{i=1}^n$ are the eigenpairs of $\mf{A}^k$. 
        \item $\lc p\lp\lambda_i\rp, \mf{v}_i\rc_{i=1}^n$ are the eigenpairs of $p\lp\mf{A}\rp$, where $p\lp \mf{A}\rp = \alpha_0\mf{I} + \alpha_1\mf{A} + \ldots + \alpha_k\mf{A}^k$.
    \end{enumerate}

    \item Prove that if $\lc\lambda_i, \mf{v}_i\rc_{i=1}^n$ are the eigenpairs of a matrix $\mf{A}$, then the eigenpairs of $\mf{A}^k$ are $\lc\lambda_i^k, \mf{v}_i\rc_{i=1}^n$. 

    \item Consider the matrices $\mf{A} = \bmx 1 & 1\\ 0 & 1 \emx$, $\mf{B} = \bmx 2 & 0 \\ 1 & 1\emx$. Are the eigenvalues of $\mf{A}\mf{B}$ equal the eigenvalues of $\mf{B}\mf{A}$?

    \item Consider the matrices $\mf{A}$ and $\mf{B}$. If $\mf{v}$ is an eigenvector $\mf{B}$, underwhat condition will $\mf{v}$ also be the eignevector of $\mf{A}\mf{B}$. Under these conditions, what will be corresponding eigenvalue of $\mf{v}$? How do your answers change in the case of $\mf{BA}$?

    \item Let $\lc \lambda_i, \mf{v}_i\rc_{i=1}^n$ are the eignepairs of a matrix $\mf{A}$. What are the eigenpairs of the folliowing?
    \begin{enumerate}
         \item $2\mf{A}$
         \item $\mf{A} - 2\mf{I}$
         \item $\mf{I} - \mf{A}$
     \end{enumerate}

     \item Let $\mf{A} = \bmx 0.6 & 0.2\\ 0.4 & 0.8\emx$. What is the value of:
     \begin{enumerate*}
         \item $A^2$
         \item $A^{100}$
         \item $A^\infty$
     \end{enumerate*}?

     \item Show that $\mf{u} \in \mb{R}^2$ is an eigenvector of $\mf{A} = \mf{u}\mf{v}^T$. What are the two eigenvalues of $\mf{A}$?

     \item Consider two similar matrices $\mf{A}$ and $\mf{B}$. Prove that the eigenvalues of $\mf{A}$ and $\mf{B}$ are the same. How are the eigenvectors of $\mf{A}$ and $\mf{B}$ related to each of other for a given eigenvalue?

     \item Find the eigenvectors of the following permutation matrix $\mf{A} = \bmx 0 & 1 & 0\\ 1 & 0 & 0 \\ 0 & 0 & 1\emx$.

     \item \textbf{Left eigenvectors}: Consider a matrix $\mf{A}$ with eigenpairs $\lc \lambda_1, \mf{v}_i\rc_{i=1}^n$. The left eigenvectors of the matrix $\mf{A}$ are the vectors that satisfy the equation, $\mf{A}^T\mf{w} = \mu \mf{w}$ (or $\mf{w}^T\mf{A} = \mu \mf{w}^T$), and let $\lc \mu_i, \mf{w}_i\rc_{i=1}^n$ be the left eigenpairs of $\mf{A}$. Show the following,
     \begin{enumerate}
         \item The eigenvalues of both $\mf{A}$ and $\mf{A}^T$ are the same.
         \item $\mf{v}_i^T\mf{w}_j = 0$. The eigenvector $\mf{v}_i$ corresponding to the eigenvalue $\lambda_i$ and the left eigvenvector $\mf{w}_j$ corresponding to the eigenvalue $\lambda_j$ are orthogonal, when $\lambda_i \neq \lambda_j$.
         \item The matrix $A$ can be expressed as a sum of rank-one matrices,
         \[ \mf{A} = \lambda_1\mf{v}_1\mf{w}_1^T + \lambda_2\mf{v}_2\mf{w}_2^T + \ldots + \lambda_n\mf{v}_n\mf{w}_n^T\]
     \end{enumerate}

     \item Prove that $\mf{A}\mf{A}^T$ has real and positive eigenvalues, and that the eigenvectors corresponding to distinct eigenvalues of $\mf{A}\mf{A}^T$ are orthogonal.

     \item If $\lc \lambda_i, \mf{v}_i\rc_{i=1}^n$ are the eigenpairs of a non-singular matrix $\mf{A}$, the prove that $\lc \lambda_i^{-1}, \mf{v}_i\rc_{i=1}^n$ are the eigenpairs of $\mf{A}^{-1}$.

     \item A matrix $\mf{A}$ is called \textit{nilpotent} if $\mf{A}^k = \mf{0}$ for some finite positive integer $k$. Prove that the $trace\lp \mf{A}\rp = 0$ for a nilpotent matrix $\mf{A}$. What are all the eigenvalues of such a matrix?

\end{enumerate}


\end{multicols}
\end{document}