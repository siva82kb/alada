% -*- root: ../assignment.tex -*-
\chapter{Orthogonality}

\begin{enumerate}[resume]
    \item Consider an orthonormal set of vectors,
    $$V = \left\{ \mf{v}_1, \mf{v}_2, \ldots \mf{v}_r\right\},\,\,\, \mf{v}_i \in \mb{R}^n\,\,\,\forall i \in \left\{1, 2, \ldots r\right\}$$
    If there is a vector $\mf{w} \in \mb{R}^n$ such that $\mf{v}_i^T\mf{w} = 0\,\,\,\forall i \in \left\{1, 2, \ldots r\right\}$. Prove that $\mf{w} \notin span\left(V\right)$.
    
    \item Consider the following set of vectors in $\mb{R}^4$.
    \[ V = \left\{
    \bmx
    1\\-2\\0\\3
    \emx,
    \bmx
    1\\1\\1\\1
    \emx,
    \bmx
    2\\-1\\1\\4
    \emx
    \right\} \]
    Find the set of all vectors that are orthogonal to $V$?

    \item For a matrix $\mf{A} \in \mb{R}^{m \times n}$, prove that $C\lp\mf{A}\rp \perp N\lp\mf{A}^T\rp$ and $C\lp\mf{A}^T\rp \perp N\lp\mf{A}\rp$.

    \item If the columns of a matrix $\mf{A} \in \mb{R}^{n \times n}$ are orthonormal, prove that $\mf{A}^{-1} = \mf{A}^T$. What is $\mf{A}^T\mf{A}$ when $\mf{A}$ is rectangular $\left(\mf{A} \in \mb{R}^{m \times n}\right)$ with orthonormal columns?

    \item What will happen when the Gram-Schmidt procedure is applied to: (a) orthonormal set of vectors; and (b) orthogonal set of vectors? If the set of vectors are columns of a matrix $\mf{A}$, then what are the corresponding $\mf{Q}$ and $\mf{R}$ matrices for the orthonormal and orthogonal cases?

    \item Consider the linear map, $\mf{y} = \mf{Ax}$, such that $\mf{x}, \mf{y} \in \mb{R}^n$ and $\mf{A} \in \mb{R}^{n \times n}$. Let us assume that $\mf{A}$ is full rank. What conditions must $\mf{A}$ satisfy for the following statements to be true,
    \begin{enumerate}
        \item $\Vert \mf{y} \Vert_2 = \Vert \mf{x} \Vert_2$, for all $\mf{x}, \mf{y}$ such that $\mf{y} = \mf{Ax}$.
        \item $\mf{y}_1^T\mf{y}_2 = \mf{x}_1^T\mf{x}_2$, for all $\mf{x}_1, \mf{x}_2, \mf{y}_1, \mf{y}_2$ such that $\mf{y}_1 = \mf{A}\mf{x}_1$ and $\mf{y}_2 = \mf{A}\mf{x}_2$. 
    \end{enumerate}
    \vspace{-0.1cm}
    \begin{color}{black!60}\small{\textbf{Note}: A linear map $\mf{A}$ with the aforementioned properties preserves lengths and angle between vectors. Such maps are encountered in rigid body mechanics.}
    \end{color}

    \item Prove that the rank of an orthogonal projection matrix $\mf{P}_{S} = \mf{UU}^T$ onto a subspace $\mc{S}$ is equal to the $\text{dim } \mc{S}$, where the columns of $\mf{U}$ form an orthonormal basis of $\mc{S}$.

    \item If the columns of $\mf{A} \in \mb{R}^{m \times n}$ represent a basis for the subspace $\mc{S} \subset \mb{R}^m$. Find the orthogonal projection matrix $\mf{P}_\mc{S}$ onto the subspace $\mc{S}$.
    
    {\color{gray} Hint: Gram-Schmidt orthogonalization.}

    \item Consider two orthogornal matrices $\mf{Q}_1$ and $\mf{Q}_2$. Is the $\mf{Q}_2^T\mf{Q}_1$ an orthogonal matrix? If yes, prove that it is so, else provide a counter-example showing $\mf{Q}_2^T\mf{Q}_1$ is not orthogonal.

    \item Let $\mf{P}_\mc{S}$ represent an orthogonal projection matrix onto to the subspace $\mc{S} \subset \mb{R}^n$. How can we obtain an orthonormal basis for $\mc{S}$ from $\mf{P}_\mc{S}$.

    \item Consider a 1 dimensional subspace spanned by the vector $\mf{u} \in \mb{R}^n$. What kind of a geometric operation does the matrix $\mf{R} = \mf{I} - 2\frac{\mf{u}\mf{u}^T}{\mf{u}^T\mf{u}}$ represent?
    
    Show that $\mf{R}$ satisfies the following properties:
    \begin{enumerate}
        \item $\mf{R}^2 = \mf{I}$
        \item Consider a vector $\mf{x} = \bmx x_1 & x_2 & \cdots & x_n \emx^\top \in \mb{R}^n$ such that $x_1 \neq 0$. If we choose $\mf{u} = $
    \end{enumerate}

    \item Prove that when a triangular matrix is orthogonal, it is diagonal.

    \item If an orthogonal matrix $\mf{Q} \in \mb{R}^{n \times n}$ is to be partitioned such that, $\mf{Q} = \bmx \mf{Q}_1 & \mf{Q}_2\emx$, then prove that $C\lp\mf{Q}_1\rp \perp C\lp\mf{Q}_2\rp$.

    \item Find an orthonormal basis for the subspace spanned by the following set,
    $$\lc \bmx1\\-1\\2\emx, \bmx-1\\-1\\-1\emx, \bmx1\\-3\\3\emx \rc$$

\end{enumerate}
