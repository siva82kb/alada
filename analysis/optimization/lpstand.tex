\documentclass{standalone}
\usepackage{tikz}
\usepackage{tikz-3dplot}

\begin{document}

\tdplotsetmaincoords{70}{80} % Set the viewing angle

\begin{tikzpicture}[scale=1.4, tdplot_main_coords]

\node[anchor=north] at (0, 1.3, 7.3) {\Huge LP in the standard (equality) form};

% Define vertices
\coordinate (A) at (5,0,0);
\coordinate (B) at (0,5,0);
\coordinate (C) at (0,0,5);
\coordinate (D) at (0,0,0);

% Draw simplex edges
\draw[thick] (A) -- (B) -- (C) -- cycle;

% Fill simplex
\filldraw[fill=gray!40, fill opacity=0.3] (A) -- (B) -- (C) -- cycle;

% Draw axes
\draw[-latex, gray] (-10,0,0) -- (10,0,0) node[anchor=north east]{\Huge{$x_1$}};
\draw[-latex, gray] (0,-4,0) -- (0,6,0) node[anchor=north west]{\Huge{$x_2$}};
\draw[-latex, gray] (0,0,-4) -- (0,0,6) node[anchor=south]{\Huge{$x_3$}};

% Add equation text
\node[anchor=north] at (2,1.85,2) {\Huge{$\tilde{\mathbf{A}}\tilde{\mathbf{x}} = \mathbf{b}$}};
\node[anchor=north] at (0, -0.3, -0.1) {\Huge{$\mathbf{0}$}};

% Draw dots at the points where the simplex crosses the axes
\coordinate (Ax) at (5,0,0);
\coordinate (By) at (0,5,0);
\coordinate (Cz) at (0,0,5);
\filldraw [black] (Ax) circle (1pt);
\filldraw [black] (By) circle (1pt);
\filldraw [black] (Cz) circle (1pt);

\end{tikzpicture}

\end{document}
